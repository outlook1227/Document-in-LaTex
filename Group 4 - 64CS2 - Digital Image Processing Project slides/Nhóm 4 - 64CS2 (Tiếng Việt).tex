\documentclass[10.5pt]{beamer}
\usepackage[utf8]{inputenc}
\usepackage[T5]{fontenc}
\usepackage{vntex}
\usepackage{times}
\usetheme{Madrid}
\usepackage{graphicx}
\usepackage{tikz}
\author{Nhóm 4 - 64CS2}
\title{Đồ án Xử lý ảnh}
\subtitle{Đề tài: Xây dựng thuật toán nhận diện khuôn mặt \\
	Giảng viên: TS Thái Thị Nguyệt}
\logo{}
\institute[HUCE]{\fontsize{9.5pt}{10.5pt}\selectfont Khoa công nghệ thông tin - Đại học Xây dựng Hà Nội}
\date{Ngày 28 tháng 5 năm 2022}
\subject{Đồ án}
\setbeamercovered{transparent}
\setbeamertemplate{navigation symbols}{\includegraphics[width=1.15cm]{FIT_HUCE.jpg}}
\AtBeginSection[]
{
	\begin{frame}
		\frametitle{\fontsize{11.5pt}{12.5pt}\selectfont Nội dung}
		\tableofcontents[currentsection]
	\end{frame}
}
\begin{document}
	\begin{frame}[plain]
		\maketitle
		
	\end{frame}
	
	% Tạo danh sách tên thành viên trong nhóm
	\begin{frame}
		\frametitle{\fontsize{11.5pt}{12.5pt}\selectfont Thành viên nhóm}
		\fontsize{8pt}{9.5pt}\selectfont Danh sách thành viên nhóm 4 - Lớp 64CS2: \\
		\begin{table}[]
			\begin{tabular}{|l|l|l|}
				\hline
				\fontsize{6.5pt}{8pt}\selectfont STT & 	\fontsize{6.5pt}{8pt}\selectfont Họ và tên  & \fontsize{6.5pt}{8pt}\selectfont Mã SV \\ \hline
					\fontsize{6.5pt}{8pt}\selectfont 1 & 	\fontsize{6.5pt}{8pt}\selectfont Nguyễn Khánh Duy &  	\fontsize{6.5pt}{8pt}\selectfont 1510964\\ 	\hline
					\fontsize{6.5pt}{8pt}\selectfont 2 & 	\fontsize{6.5pt}{8pt}\selectfont Nguyễn Đình Huy &  	\fontsize{6.5pt}{8pt}\selectfont 1524964 \\	\hline
					\fontsize{6.5pt}{8pt}\selectfont 3 & 	\fontsize{6.5pt}{8pt}\selectfont Lương Thanh Tài &  	\fontsize{6.5pt}{8pt}\selectfont 173264   \\ 	\hline
					\fontsize{6.5pt}{8pt}\selectfont 4 & 	\fontsize{6.5pt}{8pt}\selectfont Nguyễn Huy Thành &  	\fontsize{6.5pt}{8pt}\selectfont 1546864  \\ 	\hline
					\fontsize{6.5pt}{8pt}\selectfont 5 & 	\fontsize{6.5pt}{8pt}\selectfont Đỗ Đức Tiến & 		\fontsize{6.5pt}{8pt}\selectfont 1660364  \\	\hline
			\end{tabular}
		\end{table}
	\end{frame}

% Thiết lập các chương cho slide LaTex
	\begin{frame}
		\frametitle{\fontsize{11.5pt}{12.5pt}\selectfont Nội dung}
		\tableofcontents
	\end{frame}

%---------------------------------------------------------
\section{\fontsize{8.5pt}{9.5pt}\selectfont Tìm hiểu lý thuyết Xử lý ảnh}
\begin{frame}
	\frametitle{\fontsize{11.5pt}{12.5pt}\selectfont Tìm hiểu lý thuyết xử lý ảnh}
\end{frame}
%---------------------------------------------------------

%---------------------------------------------------------
\section{\fontsize{8.5pt}{9.5pt}\selectfont Cài đặt thuật toán}
\begin{frame}
	\frametitle{\fontsize{11.5pt}{12.5pt}\selectfont Cài đặt thuật toán}
		\begin{block}{\fontsize{10pt}{12.5pt}\selectfont Sơ lược thuật toán}
			\fontsize{7.5pt}{10.5pt}\selectfont Trong phần cài đặt thuật toán nhận diện khuôn mặt, thì đây là phần học có giám sát (tiếng Anh: Supervised Learning), tức là chúng ta phải dán nhãn ảnh cho khuôn mặt. Từ đó chúng ta sẽ tìm được sự khác biệt giữa các khuôn mặt và sau đó chúng ta sẽ lưu vào các model.
		\end{block}
		\begin{block}{\fontsize{10pt}{10.5pt}\selectfont Thao tác thực hiện}
			\begin{itemize}
				\fontsize{7.5pt}{10.5pt}\selectfont
				\item Thu thập khuôn mặt, rồi cắt ảnh và lưu vào các thư mục, mỗi người thì sẽ lưu vào mỗi thư mục.
				\item Chạy chương trình thì huấn luyện ảnh khuôn mặt
				\item Chọn ảnh trên thư mục và tìm kiếm khuôn mặt
				\item Nếu tìm được khuôn mặt ưng ý thì nhận diện
			\end{itemize}
		\end{block}
\end{frame}
%---------------------------------------------------------

%---------------------------------------------------------
\section{\fontsize{8.5pt}{9.5pt}\selectfont Các thuật toán chính}
\subsection{\fontsize{8.5pt}{9.5pt}\selectfont Phân vùng màu - Color Segmentation}
\subsection{\fontsize{8.5pt}{9.5pt}\selectfont Phân vùng ảnh - Image Segmentation}
\subsection{\fontsize{8.5pt}{9.5pt}\selectfont Nối ảnh - Image Matching}
\begin{frame}
	\frametitle{\fontsize{11.5pt}{12.5pt}\selectfont Các thuật toán chính}
	\fontsize{9.5pt}{11.5pt}\selectfont Trong phần này thì chúng ta sẽ tìm hiểu 3 nội dung chính trong thuật toán để thực hiện Xử lý ảnh màu:
	\begin{itemize}
		\item \fontsize{8.25pt}{9.5pt}\selectfont Phân đoạn màu sắc - Color Segmentation
		\item \fontsize{8.25pt}{9.5pt}\selectfont Phân vùng ảnh - Image Segmentation
		\item \fontsize{8.25pt}{9.5pt}\selectfont Nối ảnh - Image Matching
	\end{itemize}
	\fontsize{9.5pt}{11.5pt}\selectfont Vấn đề của Xử lý ảnh màu trong thuật toán
	\begin{itemize}
		\item \fontsize{8.25pt}{9.5pt}\selectfont Tính năng nhận diện khuôn mặt không đơn giản vì nó có rất nhiều biến thể về hình ảnh.
		\item \fontsize{8.25pt}{9.5pt}\selectfont Những hình ảnh được sử dụng trong phương pháp này có mức độ đồng nhất: tất cả các khuôn mặt đều thẳng đứng và có góc nhìn chính diện, điều kiện chiếu sáng gần như giống nhau. 
		\item \fontsize{8.25pt}{9.5pt}\selectfont Chủ yếu dựa trên các phương pháp phân đoạn màu, phân đoạn ảnh và đối sánh mẫu.
	\end{itemize}
\end{frame}
\begin{frame}{\fontsize{11.5pt}{12.5pt}\selectfont Phân đoạn màu sắc - Color Segmentation}
	\begin{block}{\fontsize{9pt}{10.5pt}\selectfont Khái quát chung}
		\fontsize{7.5pt}{10.5pt}\selectfont Mặc dù hình ảnh màu đầu vào thường ở định dạng RGB, nhưng các kỹ thuật này thường sử dụng các thành phần màu trong không gian màu, chẳng hạn như định dạng HSV hoặc YIQ. Đó là bởi vì các thành phần RGB phụ thuộc vào điều kiện ánh sáng, do đó nhận diện khuôn mặt có thể không thành công nếu điều kiện ánh sáng thay đổi. Trong số nhiều không gian màu, dự án này đã sử dụng các thành phần YCbCr vì nó là một trong các hàm MATLAB hiện có, do đó sẽ tiết kiệm thời gian tính toán. Trong không gian màu YCbCr, thông tin độ chói được chứa trong thành phần Y; và, thông tin về độ sắc bằng Cb và Cr. Do đó, thông tin về độ chói có thể được loại bỏ dễ dàng. Các thành phần RGB được chuyển đổi thành các thành phần YCbCr bằng công thức sau:
	\end{block}
	\begin{block}{}
		\begin{itemize}
			\item \fontsize{7.5pt}{10.5pt}\selectfont Phát hiện màu da trong ảnh màu là một kỹ thuật rất phổ biến và hữu ích để phát hiện khuôn mặt. Nhiều kỹ thuật đã được đưa ra để định vị các vùng màu da trong hình ảnh đầu vào. 
			\item \fontsize{7.5pt}{10.5pt}\selectfont Trong khi hình ảnh màu đầu vào thường ở định dạng RGB, các kỹ thuật này thường sử dụng các thành phần màu trong không gian màu.
		\end{itemize}
	\end{block}
\end{frame}
\begin{frame}{\fontsize{11.5pt}{12.5pt}\selectfont Phân đoạn màu sắc - Color Segmentation}
	\begin{itemize}
		\item \fontsize{7.5pt}{10.5pt}\selectfont RGB phụ thuộc vào điều kiện ánh sáng => nhận diện khuôn mặt có thể không thành công nếu điều kiện ánh sáng thay đổi.
		\item \fontsize{7.5pt}{10.5pt}\selectfont Phương pháp này đã sử dụng các thành phần YCbCr là một trong các hàm MATLAB hiện có => do đó sẽ tiết kiệm thời gian tính toán
		\item \fontsize{7.5pt}{10.5pt}\selectfont Thông tin độ chói được chứa trong thành phần Y, thông tin về tín hiệu chênh lệch màu Cb và Cr => Thông tin về độ chói có thể được loại bỏ dễ dàng
	\end{itemize}
\end{frame}
\begin{frame}{\fontsize{11.5pt}{12.5pt}\selectfont Phân đoạn màu sắc - Color Segmentation}
	\begin{itemize}
		\item \fontsize{7.5pt}{10.5pt}\selectfont RGB phụ thuộc vào điều kiện ánh sáng => nhận diện khuôn mặt có thể không thành công nếu điều kiện ánh sáng thay đổi.
		\item \fontsize{7.5pt}{10.5pt}\selectfont Phương pháp này đã sử dụng các thành phần YCbCr là một trong các hàm MATLAB hiện có \makebox[0.5cm]{\[\Rightarrow\]} do đó sẽ tiết kiệm thời gian tính toán
		\item \fontsize{7.5pt}{10.5pt}\selectfont Thông tin độ chói được chứa trong thành phần Y, thông tin về tín hiệu chênh lệch màu Cb và Cr => Thông tin về độ chói có thể được loại bỏ dễ dàng
	\end{itemize}
	\begin{block} {\fontsize{9pt}{10.5pt}\selectfont Công thức chuyển đổi RGB - YCbCr: }
	\fontsize{6.5pt}{7.5pt}\selectfont
		\[Y = 0.229R + 0.587G + 0.114B\]
		\[Cb =  - 0.169R - 0.332G + 0.500B\]
		\[Cr = 0.500R - 0.419G - 0.081B\]
	\end{block}
\end{frame}
%---------------------------------------------------------

%---------------------------------------------------------
\section{\fontsize{8.5pt}{9.5pt}\selectfont Cài đặt giao diện}
\subsection{\fontsize{8.5pt}{9.5pt}\selectfont Cài đặt Tkinter}
\subsection{\fontsize{8.5pt}{9.5pt}\selectfont Cài đặt PyQt5}
\begin{frame}
	\frametitle{\fontsize{11.5pt}{12.5pt}\selectfont Cài đặt giao diện}
\end{frame}
%---------------------------------------------------------

%---------------------------------------------------------
\section{\fontsize{8.5pt}{9.5pt}\selectfont Đánh giá bài toán}
\begin{frame}
	\frametitle{\fontsize{11.5pt}{12.5pt}\selectfont Đánh giá bài toán}
\end{frame}
%---------------------------------------------------------
%---------------------------------------------------------
\section{\fontsize{8.5pt}{9.5pt}\selectfont Tài liệu tham khảo}
\begin{frame}
	\frametitle{\fontsize{11.5pt}{12.5pt}\selectfont Tài liệu tham khảo}
\end{frame}
%---------------------------------------------------------

\end{document}
